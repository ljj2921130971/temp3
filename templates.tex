% !Mode:: "laTeX:UTF-8"
%!TEX program  = xelatex
%采用封面
\documentclass[no-math,bwprint]{YangThesis} 
\usepackage{amsmath, amsfonts, amssymb} % 数学公式、符号

%导入设置文件
%设置参数均存储在settings.tex文件中,
%添加或修改需要在该文件中进行
%settings.tex
%YangTemplate 的配置文件
%采用input放入主文档中,不可单独编译

%采用Times New Roman字体
\usepackage[T1]{fontenc}
\usepackage{newtxtext, newtxmath}
\usepackage{float}
\usepackage{cite}

%定制tikz流程图形状
\usepackage{tikz}
\usetikzlibrary{shapes.geometric, arrows}
\tikzstyle{startstop} = [rectangle, rounded corners, minimum width = 2cm, minimum height=1cm,text centered, draw = black]
\tikzstyle{io} = [trapezium, trapezium left angle=70, trapezium right angle=110, minimum width=2cm, minimum height=1cm, text centered, draw=black]
\tikzstyle{process} = [rectangle, minimum width=3cm, minimum height=1cm, text centered, draw=black]
\tikzstyle{decision} = [diamond, aspect = 3, text centered, draw=black]
% 箭头形式
\tikzstyle{arrow} = [->,>=stealth]

%设置编号格式
\newcounter{rowno}
\numberwithin{equation}{section}
\numberwithin{figure}{section}
\numberwithin{table}{section}
\renewcommand{\thefigure}{\arabic{section}-\arabic{figure}}
\renewcommand{\thetable}{\arabic{section}-\arabic{table}}
\renewcommand{\theequation}{\arabic{section}-\arabic{equation}}

%新数学命令
\newcommand\dif{\mathrm{d}}
\newcommand\no{\noindent}
\newcommand\dis{\displaystyle}
\newcommand\ls{\leqslant}
\newcommand\gs{\geqslant}

\newcommand\limit{\dis\lim\limits}
\newcommand\limn{\dis\lim\limits_{n\to\infty}}
\newcommand\limxz{\dis\lim\limits_{x\to0}}
\newcommand\limxi{\dis\lim\limits_{x\to\infty}}
\newcommand\limxpi{\dis\lim\limits_{x\to+\infty}}
\newcommand\limxni{\dis\lim\limits_{x\to-\infty}}

%默认上限为无穷
\newcommand\sumnf{\dis\sum\limits_{n=1}^{\infty}}
\newcommand\sumnz{\dis\sum\limits_{n=0}^{\infty}}
\newcommand\sumkf{\dis\sum\limits_{k=1}^{\infty}}
\newcommand\sumkz{\dis\sum\limits_{k=0}^{\infty}}
\newcommand\sumifn{\dis\sum\limits_{i=1}^{n}}
\newcommand\sumizn{\dis\sum\limits_{i=0}^{n}}
\newcommand\sumkzn{\dis\sum\limits_{k=0}^n}
\newcommand\sumkfn{\dis\sum\limits_{k=1}^n}

\newcommand\pzx{\dis\frac{\partial z}{\partial x}}
\newcommand\pzy{\dis\frac{\partial z}{\partial y}}

\newcommand\pfx{\dis\frac{\partial f}{\partial x}}
\newcommand\pfy{\dis\frac{\partial f}{\partial x}}

\newcommand\pzxx{\dis\frac{\partial^2 z}{\partial x^2}}
\newcommand\pzxy{\dis\frac{\partial^2 z}{\partial x\partial y}}
\newcommand\pzyx{\dis\frac{\partial^2 z}{\partial y\partial x}}
\newcommand\pzyy{\dis\frac{\partial^2 z}{\partial y^2}}

\newcommand\pfxx{\dis\frac{\partial^2 f}{\partial x^2}}
\newcommand\pfxy{\dis\frac{\partial^2 f}{\partial x\partial y}}
\newcommand\pfyx{\dis\frac{\partial^2 f}{\partial y\partial x}}
\newcommand\pfyy{\dis\frac{\partial^2 f}{\partial y^2}}

\newcommand\intzi{\dis\int_{0}^{+\infty}}
\newcommand\intd{\dis\int}
\newcommand\intab{\dis\int_a^b}

\newenvironment{mfrac}[2]%
{\raise0.5ex\hbox{$#1$}\! \left/ \! \lower0.5ex\hbox{$#2$}\right.}

%定义新数学符号
\DeclareMathOperator{\sgn}{sgn}
\DeclareMathOperator{\arccot}{arccot}
\DeclareMathOperator{\arccosh}{arccosh}
\DeclareMathOperator{\arcsinh}{arcsinh}
\DeclareMathOperator{\arctanh}{arctanh}
\DeclareMathOperator{\arccoth}{arccoth}
\DeclareMathOperator{\grad}{\bf{grad}}

\usepackage{url}

%自定义字号大小命令
%修改18pt为想要的字号即可
\newcommand{\myfont}{\fontsize{18pt}{\baselineskip}\selectfont}

%定制页眉页脚
\usepackage{fancyhdr}
\pagestyle{fancy}
%页眉
\lhead{}
\chead{黎曼Zeta函数的可视化}
\rhead{}
%页脚
\lfoot{}
\cfoot{-\thepage-}
\rfoot{}

%页眉页脚横线
\renewcommand{\headrulewidth}{0.4pt}
%\renewcommand{\footrulewidth}{0pt}

% %双横线
\newcommand{\makeheadrule}{%
\makebox[0pt][l]{\rule[0.2\baselineskip]{\headwidth}{1.3pt}}%
\rule[0.35\baselineskip]{\headwidth}{2.5pt}}
\renewcommand{\headrule}{%
{\if@fancyplain\let\headrulewidth\plainheadrulewidth\fi
\makeheadrule}}
\makeatother

%设置脚注编号格式
\renewcommand{\thefootnote}{\fnsymbol{footnote}}

% 封面信息
\papercategory{ \textbf{毕业论文(设计)}}
\title{\textbf{题目:黎曼Zeta函数的可视化}}
\schoolname{马龙}
\departname{软件学院}
\classnumber{软件工程}
\authorname{罗俊杰}
\studentID{201800301273}
\dateinput{\today}


%开始写文章
\begin{document}

%生成标题
  \maketitle
 %页码从1开始计数
 \setcounter{page}{1}
 %页码采用罗马数字格式
 \pagenumbering{Roman}
 
 %开始写中文摘要
 \begin{abstract}
 %将摘要添加到目录中
 \addcontentsline{toc}{section}{摘要}
 黎曼Zeta函数$\zeta(z)$是定义在复数平面上的一个解析函数,它的表达式是正整数的$z$次幂的倒数之和,
 它的零点分布反映了素数的分布规律,因而在在解析数论中有着重要研究价值。本设计利用Python对黎曼Zeta函数进行了多种可视化,
 并且重点研究了临界线$\Re(z) = 1 / 2 $。

%关键词
\keywords{Python;可视化;黎曼Zeta函数}
\end{abstract}
%开始写英文摘要
\begin{abstracten}
	\addcontentsline{toc}{section}{Abstract}
 
	Riemann Zeta Function $\zeta(z)$ is an analytic function defined on the complex plane. 
	Its expression is the sum of the reciprocal of the power of $z$ of a positive integer.
	Its zero point distribution reflects the distribution law of prime numbers, 
	so it has important research value in analytic number theory. This design uses Python to visualize Riemann Zeta Function,
	and focus on the critical boundary $\Re(z) = 1 / 2 $.
	%关键词
	\keywordsen{Python;Visualization;Riemann Zeta Function}
\end{abstracten}	
%强制目录二字位于最上方
\vspace{-1.3cm}
%生成目录
\tableofcontents
%将目录项添加到目录中
% \addcontentsline{toc}{section}{目录}

%撰写正文
\clearpage
%调整标题与上边的距离
\vspace{-1cm}
%第1章的标题
\section{绪论}
%页码从1开始计数
\setcounter{page}{1}
%页码格式采用阿拉伯数字
\pagenumbering{arabic}
\subsection{国内外研究背景}
%正文内容,注意LaTeX分段由两种方法,直接空一行或者使用\par
%默认首行缩进,不需要在这里手动敲空格,已有宏包处理
随着社
\subsection{深度学习与纳米光子学芯片}

\subsection{本文主要内容}
%需手动分页,为实现分页与不分页均可
\newpage
\section{系统框架}

\subsection{深度学习算法}

近年来,基于
\subsection{光学实现}
在机器\textsuperscript{\cite{ShenDeep}}

\newpage
\section{神经网络算法}

\subsection{基于复值}

\subsubsection{实值}

人
\subsubsection{复值神}

复值
\subsection{对具}

{\bfseries\song1.对任务一的分析和对策}
公式示例:Equation (\ref{eq1}) is an integration.

\begin{equation}\label{eq1}
\begin{split}
\int_0^1 x^2 \dif x = \frac{1}{3}\\
\iint_\sigma xy\dif x \dif y = \Theta
\end{split}
\end{equation}

\begin{equation}
\sumifn \frac{1}{i^2}=\frac{n(n+1)(2n+1)}{6}
\end{equation}

\newpage
\section{模型的假设}

\subsection{符号说明}
\begin{enumerate}[label=\arabic*.]
\item 假设CNC在不工作时不会发生故障 
\end{enumerate}



\newpage
\section{结论}
花括号里的内容为此条参考文献的标签,花括号里的内容为此条参考文献的标签,花括号里的内容为此条参考文献的标签,花括号里的内容为此条参考文献的标签,花括号里的内容为此条参考文献的标签,

\newpage

\addcontentsline{toc}{section}{致谢}
\LARGE\textbf{致谢\\}
\large{花括号里的内容为此条参考文献的标签,花括号里的内容为此条参考文献的标签,花括号里的内容为此条参考文献的标签,花括号里的内容为此条参考文献的标签,花括号里的内容为此条参考文献的标签,
}
\newpage
\begin{thebibliography}{9}
	\addcontentsline{toc}{section}{参考文献}
	
	% 每条参考文献均需用<\bibitem{}>引出,
	% 花括号里的内容为此条参考文献的标签,
	% 可用<\cite{}>引用。
	% 若参考文献条数超过99,
	% 请联系作者修改.cls类文件,
	% 否则空格略有不完美。
	\bibitem{ShenDeep} Shen Y , Harris N C , Skirlo S , et al. Deep learning with coherent nanophotonic circuits[J]. Nature Photonics, 2017.

\end{thebibliography}
\newpage
\LARGE{附录一:\\}
\addcontentsline{toc}{section}{附录}
花括号里的内容为此条参考文献的标签,花括号里的内容为此条参考文献的标签,花括号里的内容为此条参考文献的标签,花括号里的内容为此条参考文献的标签,花括号里的内容为此条参考文献的标签,

\end{document} 